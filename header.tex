% !TeX encoding = UTF-8
% !TeX program = lualatex
% !TeX spellcheck = en_US
% !TeX root = thesis.tex


%%%%%%%%%%%%%%%%%%%%%%%%%%%%%%%%%%%%%
%BASIC OPTIONS
%%%%%%%%%%%%%%%%%%%%%%%%%%%%%%%%%%%%%


\documentclass[
%draft,
12pt,
%oneside,
twoside,
paper=a4,									% Seitengröße%
DIV=9,										% Satzspiegel: z.B. calc oder 6-15 //war ursprünglich 13
BCOR=4mm, 									% Bindekorrektur
%listof=chapterentry,						% Im Abbildungsverzeichnis und Tabellenverzeichnis auch Kapitel anzeigen
toc=bibliography,							% Literaturverzeichnis mit ins Inhaltsverzeichnis eintragen
toc=listof,
footnotes=multiple,							% mehrere Fußnoten separieren
%parindent,									% Absatzstil hier: Einrückung;
parskip=half,								% Absatzstil hier: keine Einrückung;
numbers=noenddot,							% Keine Punkte nach Nummerierung
]{scrbook}


%%%%%%%%%%%%%%%%%%%%%%%%%%%%%%%%%%%%%
%TITLEPAGE
%%%%%%%%%%%%%%%%%%%%%%%%%%%%%%%%%%%%%
%%%%%%%%%%%%%%%%%%%%%%%%%%%%%%%%%%%%%
%TITLEPAGE
%%%%%%%%%%%%%%%%%%%%%%%%%%%%%%%%%%%%%

\newcommand{\titel}{Customizing OpenFOAM to assess\\ wind-induced natural ventilation\\ potential of classrooms\newline {\Large A case study for BRAC University}}
\newcommand{\untertitel}{}
\newcommand{\HRule}{\rule{\linewidth}{0.5mm}} % New command to make the lines in the title page
\newcommand{\art}{Master's Thesis}
\newcommand{\fachgebiet}{}
\newcommand{\autor}{Patrick Kastner}
\newcommand{\wohnort}{81369 München}
\newcommand{\univname}{Technische Universität München}
\newcommand{\studienbereich}{Energy-efficient and Sustainable Building}
\newcommand{\groupname}{}
\newcommand{\deptname}{}
\newcommand{\degreename}{Master of Science (MSc)}
\newcommand{\matrikelnr}{03650879}
\newcommand{\erstgutachter}{Prof. Dipl.-Ing. Thomas Auer}
\newcommand{\zweitgutachter}{Dr. rer. nat. Wolfgang Kessling}
\newcommand{\drittgutachter}{Markus Krauß, M.Eng.}
\newcommand{\viertgutachter}{Dipl.-Ing. Christian Oberdorf}
\newcommand{\ort}{München}

%%%%%%%%%%%%%%%%%%%%%%%%%%%%%%%%%%%%%
%CUSTOM COMMANDS
%%%%%%%%%%%%%%%%%%%%%%%%%%%%%%%%%%%%%

\newcommand{\up}[2]{#1\textsuperscript{#2}}    % z.B. \up I{fdfgdfs}
\newcommand{\down}[2]{#1\textsubscript{#2}}

\newcommand{\OF}{\mbox{OpenFOAM}}
\newcommand{\CR}{\gls{CR}}
\newcommand{\WT}{\gls{WT}}
\newcommand{\BC}{\gls{BC}}
\newcommand{\ABL}{\gls{ABL}}
\newcommand{\NVP}{\gls{NVP}}
\newcommand{\USED}{{\color{red}{USED}}}
\newcommand{\BRAC}{BRAC University}
\newcommand{\CFD}{\gls{CFD}}
\newcommand{\ACR}{\gls{ACR}}
\newcommand{\kEps}{$k-\varepsilon$}
\newcommand{\kOmega}{$k-\omega$}
\newcommand{\R}{\gls{R}}
\newcommand{\COtwo}{CO$_2$}
\newcommand{\yplus}{\gls{symb:yplus}}
\newcommand{\yPlus}{\gls{symb:yplus}}
\newcommand{\SD}{\gls{SD}}


%%%%%%%%%%%%%%%%%%%%%%%%%%%%%%%%%%%%%
%FONTS
%%%%%%%%%%%%%%%%%%%%%%%%%%%%%%%%%%%%%

\usepackage{textcomp}
\usepackage[]{euler}
\usepackage{fontspec}
\usepackage{luatextra}
\usepackage{csquotes}

%FONT Combination 1

\setmainfont{Erewhon}[
Extension=.otf,
UprightFont=*-Regular,
ItalicFont=*-Italic,
BoldFont=*-Bold,
BoldItalicFont=*-BoldItalic,
SlantedFont=*-RegularSlanted,
BoldSlantedFont=*-BoldSlanted,
]
\setsansfont{texgyreheros}[
Scale=MatchLowercase,% or MatchUppercase
Extension=.otf,
UprightFont=*-regular,
ItalicFont=*-italic,
BoldFont=*-bold,
BoldItalicFont=*-bolditalic,
]

%FONT Combination 2
\setmainfont{XCharter-Roman}
\setsansfont{texgyreheros}[
Scale=MatchLowercase,% or MatchUppercase
Extension=.otf,
UprightFont=*-regular,
ItalicFont=*-italic,
BoldFont=*-bold,
BoldItalicFont=*-bolditalic,
]

%FONT Combination 3 (Adobe Fonts)

%\setmainfont{MinionPro-Regular.otf}[
%ExternalLocation=./fonts/,
%Path=fonts/,
%Ligatures=TeX,
%Numbers=OldStyle,
%SlantedFont = MinionPro-Regular.otf,
%BoldFont = MinionPro-Bold.otf ]
%\setsansfont{MyriadPro-Regular.otf}[
%ExternalLocation=./fonts/,
%Path=fonts/,
%SlantedFont = MyriadPro-It.otf,
%BoldFont = MyriadPro-Bold.otf ]
%\setmonofont{CourierStd.otf}[
%ExternalLocation=./fonts/,
%Path=fonts/,]

%LEAVE THIS UNTOUCHED FOR NOW


%%%\setmathfont{XITS Math}
%%%\setmathfont[range=\mathup/{num,latin,Latin,greek,Greek}]{Minion Pro}
%%%\setmathfont[range=\mathbfup/{num,latin,Latin,greek,Greek}]{MinionPro-Bold}
%%%\setmathfont[range=\mathit/{num,latin,Latin,greek,Greek}]{MinionPro-It}
%%%\setmathfont[range=\mathbfit/{num,latin,Latin,greek,Greek}]{MinionPro-BoldIt}
%%%\setmathfont[range=\mathscr,StylisticSet={1}]{XITS Math}
%%%\setmathfont[range={"005B,"005D,"0028,"0029,"007B,"007D,"2211,"002F,"2215 } ]{Latin Modern Math} % brackets, sum, /
%%%\setmathfont[range={"002B,"002D,"003A-"003E} ]{MnSymbol} % + - < = >
%%%\setmathrm{Minion Pro}
%%%
%%%\usepackage{luaotfload,lualatex-math}


%%%%%%%%%%%%%%%%%%%%%%%%%%%%%%%%%%%%%
%FONT MISC SETTINGS
%%%%%%%%%%%%%%%%%%%%%%%%%%%%%%%%%%%%%


\RedeclareSectionCommands[ %Linebreak after paragraphs
afterskip=1sp
]{paragraph,subparagraph}
\usepackage{etex}
\reserveinserts{20}
\usepackage[english]{babel}				% Schriftsatzerweiterung
\usepackage{alphabeta} % Use greek letters as text and in PDF bookmarks
\usepackage{bm}								% Enable bold math symbols
\usepackage{microtype}%
%\usepackage{geometry}

\makeatletter
\g@addto@macro\bfseries{\boldmath} % bold math in section heading
\makeatother

\usepackage{wrapfig}					% Von Text umflossene Grafiken
\usepackage{subfig}



\usepackage{ellipsis} % Weißraum bei Ellipsen optimal
\usepackage[NewCommands]{ragged2e} % Flattersatz mit (!) Silbentrennung
\clubpenalty = 10000 % Schusterjungen und Hurenkinder vermeiden
\widowpenalty = 10000 % Schusterjungen und Hurenkinder vermeiden
\displaywidowpenalty = 10000 % Schusterjungen und Hurenkinder vermeiden


\usepackage{marvosym}						% Euro-Symbol verfügbar machen
\usepackage[official,right]{eurosym}


%%%%%%%%%%%%%%%%%%%%%%%%%%%%%%%%%%%%%
%GRAPHICS
%%%%%%%%%%%%%%%%%%%%%%%%%%%%%%%%%%%%%

\usepackage{graphicx} 					% Grafiken einbinden
\makeatletter							% Der folgende neue Befehl bindet Bilder in der Originalgrösse ein, falls sie weniger breit als die Seite sind. Sonst wird das Bild auf Seitenbreite skaliert.
\def\ScaleIfNeeded{%
	\ifdim\Gin@nat@width>\linewidth
	\linewidth
	\else
	\Gin@nat@width
	\fi
}
\makeatother


%%%%%%%%%%%%%%%%%%%%%%%%%%%%%%%%%%%%%
%COLORS
%%%%%%%%%%%%%%%%%%%%%%%%%%%%%%%%%%%%%


\usepackage[dvipsnames]{xcolor}
\definecolor{airforceblue}{rgb}{0.36, 0.54, 0.66}
\definecolor{royalblue}{rgb}{0.0, 0.14, 0.4}
\definecolor{tumblue}{RGB}{15 , 27 , 95}
\definecolor{tumblue}{RGB}{15 , 27 , 95}
\definecolor{lightblue}{RGB}{220 , 230 , 241}
\definecolor{tum_blue}{RGB}{0, 101, 189}
\definecolor{tum_white}{RGB}{255 , 255 , 255}
\definecolor{tum_black}{RGB}{0,0,0}
%Secondary
\definecolor{tum_blue2}{RGB}{0 , 82 , 147}
\definecolor{tum_blue3}{RGB}{0 , 51 , 89}
\definecolor{tum_grey1}{RGB}{088 , 088 , 090}
\definecolor{tum_grey2}{RGB}{156 , 157 , 159}
\definecolor{tum_grey3}{RGB}{217 , 218 , 219}
%Highlights
\definecolor{tum_beige}{RGB}{218 , 215 , 203}
\definecolor{tum_orange}{RGB}{227 , 114 , 34}
\definecolor{tum_green}{RGB}{162 , 173 , 0}
\definecolor{tum_lightblue}{RGB}{152 , 198 , 234}
\definecolor{tum_turquoise}{RGB}{100 , 160 , 200}
\definecolor{darkred}{rgb}{0.7, 0.11, 0.11}



%%%%%%%%%%%%%%%%%%%%%%%%%%%%%%%%%%%%%
%URLs
%%%%%%%%%%%%%%%%%%%%%%%%%%%%%%%%%%%%%


%\KOMAoptions{DIV=last}					% Anpassung des Satzspiegels an Schriften, (nur bei DIV=calc wirksam)
\PassOptionsToPackage{hyphens}{url}
%\usepackage[hyphenbreaks]{breakurl}
\usepackage[hyphens]{url}
%\urlstyle{same}
\makeatletter
\g@addto@macro{\UrlBreaks}{\UrlOrds}
\makeatother


%URL smaller
\usepackage{relsize}
\renewcommand*{\UrlFont}{\ttfamily\smaller\relax}


%%%%%%%%%%%%%%%%%%%%%%%%%%%%%%%%%%%%%
%HYPERREF
%%%%%%%%%%%%%%%%%%%%%%%%%%%%%%%%%%%%%

%\usepackage[hang, flushmargin]{footmisc}
\usepackage[breaklinks, hidelinks,hyperfootnotes=false, unicode]{hyperref} %Clickable hyperlinks everywhere
\def\UrlBigBreaks{\do\/\do-\do:}
\hypersetup{
	draft=false,
	%    bookmarks=true,
	bookmarksopen=true,
	bookmarksopenlevel=1,  %hier darf nur ne Zahl stehen, ansonsten produziert TeX nen Fehler in Zusammenhang mit 		bookmarksopen
	bookmarksdepth=4, %deprecated
	bookmarksnumbered=true,
	linktocpage=true,       %break links correctly in listoftables/figures
	unicode=true,          % non-Latin characters in Acrobat’s bookmarks
	pdftoolbar=true,        % show Acrobat’s toolbar?
	pdfmenubar=true,        % show Acrobat’s menu?
	pdffitwindow=true,     % window fit to page when opened
	pdfstartview={XYZ null null 1.00},
	%         XYZ 	left top zoom   Sets a coordinate and a zoom factor. If any one is null, the source link value is used. null null null will give the same values as the current page.   Fit   Fits the page to the window.    FitH 	top
	%    Fits the width of the page to the window.    FitV 	left     Fits the height of the page to the window.    FitR 	left bottom right top    Fits the rectangle specified by the four coordinates to the window.    FitB  Fits the page bounding box to the window.
	%    FitBH 	top  Fits the width of the page bounding box to the window.
	%    FitBV 	left  Fits the height of the page bounding box to the window.
	pdftitle = {\titel},    % title
	pdfauthor = {\autor},     % author
	pdfsubject = {\art},   % subject of the document
	pdfcreator={\autor},   % creator of the document
	pdfproducer={\autor}, % producer of the document
	pdfkeywords= {} {} {} {}, % list of keywords
	pdfnewwindow=true,      % links in new window
	pdfpagelayout={TwoColumnRight},
	colorlinks=true,       % false: boxed links; true: colored links
	linkcolor=black,          % color of internal links
	citecolor=tumblue,        % color of links to bibliography
	filecolor=black,      % color of file links
	urlcolor=tumblue,           % color of external links
	anchorcolor =black,
	linkbordercolor={blue!35!black},          % color of internal links
	citebordercolor={blue!35!black},        % color of links to bibliography
	filebordercolor={blue!35!black},      % color of file links
	urlbordercolor={blue!35!black},           % color of external links
	menucolor =red,
	runcolor =cyan,
	pdfencoding=auto,
}


\usepackage{bookmark}
\usepackage{xspace}				% Fixes usage of spaces
%\usepackage{layout}						% Layout überprüfen
\usepackage{setspace} 	% Linespacing: singelspacing, onehalfspacing, doublespacing
\usepackage[english, plain]{fancyref} %Cross-referencing
\usepackage{upgreek}
\usepackage{lscape}						% Landscpae Seiten
\usepackage{pdflscape}					% Landscape Seiten drehen
\usepackage{gensymb}					% Celsius anzeigen
\usepackage{subfig}					% mehrere Bilder pro Zeile
\usepackage{pdfpages}
\usepackage[footnote]{acronym}			% Abkürzungsverzeichnis
\usepackage{mathtools} % braces in math environment
\usepackage{epigraph}
%\usepackage{pgfplots}
%\usepackage{pgfkeys}
\usepackage{calc}
\usepackage{tikz}
\usepackage{setspace}
%\usepackage[version=3]{mhchem}			%chemische Formeln
\usepackage[%format=hang
,labelfont=bf,
%tableposition=top %no effefct with KOMA class
]{caption} %Bildunterschriften kleiner
\renewcommand\captionfont{\footnotesize\sffamily} % Bildtitel umformatieren
\usepackage[rightcaption]{sidecap}	%caption zentr./seitl.
\makeatletter
\newenvironment{sidefigure}{\SC@float[c]{figure}}{\endSC@float} %caption zentr./seitl.
\makeatother
\makeatletter
\newenvironment{sidetable}{\SC@float[c]{table}}{\endSC@float} %caption zentr./seitl.
\makeatother


%%%%%%%%%%%%%%%%%%%%%%%%%%%%%%%%%%%%%
%UNITS
%%%%%%%%%%%%%%%%%%%%%%%%%%%%%%%%%%%%%


\usepackage[
separate-uncertainty  = true,
uncertainty-separator =  {\,},
mode = text,
output-decimal-marker ={.},
multi-part-units      = single,
range-units           = single,
%range-phrase          = {--},
]{siunitx} 		%SI Einheit
\sisetup{
	%list-final-separator = { \translate{und} },
	%range-phrase = { \translate{bis} },
	%list-pair-separator = { \translate{und} },
	%exponent-product = \cdot
	%detect-all, %apply document fonts for siunitx
	%math-rm=\mathsf,
	%text-rm=\sffamily,
	%    locale                  = US,
	%input-decimal-markers   = {.},
	%output-decimal-marker   = {.},
	%input-ignore            = {,},
	%group-digits            = true,
	%group-separator         = {,},
	%group-separator         = {},
	%tight-spacing           = true,
	%input-signs             = ,
	%input-symbols           = ,
	%input-open-uncertainty  = ,
	%input-close-uncertainty = ,
	table-align-text-pre    = false,
}
%round-mode              = figures, %places
%round-precision         = 3,
%round-integer-to-decimal= false,
%zero-decimal-to-integer= false,
%add-decimal-zero = false,
%add-integer-zero = false,

%table-space-text-pre    = (,
%table-space-text-post   = ),

\usepackage{textpos}

%%%%%%%%%%%%%%%%%%%%%%%%%%%%%%%%%%%%%
%TABLES
%%%%%%%%%%%%%%%%%%%%%%%%%%%%%%%%%%%%%

\usepackage{booktabs} % Lines betwenn tables
\usepackage{longtable}					% Tables that are longer than one page
\usepackage{tablefootnote}				% Footnote in tables
\usepackage{tabularx, longtable, array} % Tabellen
\usepackage{ltablex} %advanced longtables across multiple pages
\usepackage{ltxtable}
\usepackage{floatrow}
\floatsetup[table]{capposition=top}
\usepackage{etoolbox} %Change font of tables
\usepackage{multirow} % merge rown in tables
\usepackage{color, colortbl}	%You will need the following two packages, the first to define new colors and the latter to actually color the table
\usepackage[para,online,flushleft]{threeparttable}	 % Footnotes in tables

% San serif table font

%\AtBeginEnvironment{tabular}{\rmfamily}

%%%%%%%%%%%%%%%%%%%%%%%%%%%%%%%%%%%%%
%TABLE WITH SMALL FONT
%%%%%%%%%%%%%%%%%%%%%%%%%%%%%%%%%%%%%


\newenvironment{tabularsmall}{%
	\fontsize{8}{12}\selectfont\tabular
}{%
	\endtabular
}


%%%%%%%%%%%%%%%%%%%%%%%%%%%%%%%%%%%%%
%CAPTIONS
%%%%%%%%%%%%%%%%%%%%%%%%%%%%%%%%%%%%%

\usepackage{caption}
%\usepackage{subcaption}
\captionsetup[subfloat]{font=sf,size=footnotesize}
\usepackage{sidecap} %captions on the side of figures


%Captions within pdfpages
\makeatletter
\newcommand*{\AM@pagecommandstar}{}
\define@key{pdfpages}{pagecommand*}{\def\AM@pagecommandstar{#1}}
\patchcmd{\AM@output}{\begingroup\AM@pagecommand\endgroup}
{\ifthenelse{\boolean{AM@firstpage}}{\begingroup\AM@pagecommandstar\endgroup}{\begingroup\AM@pagecommand\endgroup}}{}{} % Patch to use new option
\patchcmd{\AM@split@optionsii}{\equal{pagecommand}{\AM@temp}\or}
{\equal{pagecommand}{\AM@temp}\or\equal{pagecommand*}{\AM@temp}\or}{}{}
\makeatother
%Captions within pdfpages


% Kommando zum Ausbügeln des Bugs "\subfloat ohne \caption"
\makeatletter
\providecommand\phantomcaption{\caption@refstepcounter\@captype}
\makeatother


\usepackage[]{hypcap}  %Links to image directly, not to caption


%%%%%%%%%%%%%%%%%%%%%%%%%%%%%%%%%%%%%
%MISC PACKAGES
%%%%%%%%%%%%%%%%%%%%%%%%%%%%%%%%%%%%%


\usepackage{blindtext}
\usepackage{enumitem}

%Itemize indention
\setlist[itemize,1]{leftmargin=\dimexpr 26pt-.22in}
%\setitemize{noitemsep,topsep=0pt,parsep=0pt,partopsep=0pt} % Change size of itemize environments

\usepackage{listings}
\usepackage{morewrites}
%  Multiple Columns
\usepackage{multicol}
\usepackage{emptypage} %removes headers and footers on empty pages.
%% Custom commands

\usepackage{soul}

%% New itemize environment for equations

\newenvironment{equationitem}
{ \begin{itemize}
		\setlength{\itemsep}{1pt}
		\setlength{\parskip}{1pt}
		\setlength{\parsep}{1pt}}
	{ \end{itemize}     }


%% Appendix in toc

\usepackage[
%  toc
]{appendix}

\usepackage[
% obeyDraft
]{todonotes}

%% Counters for figures and tables

\usepackage[figure,table,equation]{totalcount}


\usepackage{placeins} %clear floats without new page



%%%%%%%%%%%%%%%%%%%%%%%%%%%%%%%%%%%%%
%BIBLIOGRAPHY
%%%%%%%%%%%%%%%%%%%%%%%%%%%%%%%%%%%%%

\usepackage[compress,sort,
numbers
]{natbib}					% Naturwissenschaftlicher Zitierstil
%\bibliographystyle{agsm}
\bibliographystyle{dinat}
%\bibliographystyle{newapa}
\renewcommand*{\bibfont}{\sffamily}

%% Change Bibliography Header


\makeatletter
\renewcommand\bibsection{%
	%	\chapter*{{\sffamily\huge\bibname}\@mkboth{\sffamily\MakeUppercase{\bibname}}{\sffamily\bibname}}%
	\chapter*{{\sffamily\huge\bibname}\@mkboth{\sffamily\bibname}{\sffamily\bibname}}%
}%
\makeatother




%%% APACITE

%\usepackage[%nodoi,
%nosectionbib,numberedbib]{apacite}
%\bibliographystyle{apacite}
%\AtBeginDocument{\urlstyle{APACsame}}

%\renewcommand\bibliographytypesize{\small}
%\renewcommand\bibliographystyle{apacite}



%Backlinks vom LitVerz zu den Seiten
% #1: number of distinct back references
% #2: backref list with distinct entries
% #3: number of back references including duplicates
% #4: backref list including duplicates
\RequirePackage[hyperpageref]{backref}
\renewcommand{\backreflastsep}{ and~}
\renewcommand{\backreftwosep}{ and~}
\renewcommand{\backref}[1]{}% for backref < 1.33 necessary
\renewcommand{\backrefalt}[4]{
	\ifnum#1=0
	%No cited.
	\else
	\ifnum#1=1
	\footnotesize \mbox{(cited on page #2)}
	\else
	%\footnotesize \mbox{{\color{darkred}(cited on pages #2)}}
	\footnotesize \mbox{(cited on pages #2)}
	\fi
	\fi
}


%%%%%%%%%%%%%%%%%%%%%%%%%%%%%%%%%%%%%
%GLOSSARIES
%%%%%%%%%%%%%%%%%%%%%%%%%%%%%%%%%%%%%

\usepackage[
nomain,
nonumberlist,
acronym,
%section
]
{glossaries-extra}

\glsaccessname{unit}
\setlength{\glsdescwidth}{15cm}
\glssetcategoryattribute{acronym}{glossdescfont}{textsf}
\glssetcategoryattribute{acronym}{glossnamefont}{textsf}
\setabbreviationstyle[acronym]{long-short}

\glssetnoexpandfield{unit}

\newglossarystyle{symbunitlong}{%
	\vspace*{.3cm}
	\setglossarystyle{long3col}% base this style on the list style
	\renewenvironment{theglossary}{% Change the table type --> 3 columns
		\begin{longtable}{lp{0.6\glsdescwidth}>{\arraybackslash}p{2cm}}}%
		{\end{longtable}}%
	%
	\renewcommand*{\glossaryheader}{%  Change the table header
		\sffamily\bfseries Sign & \sffamily\bfseries Description & \sffamily\bfseries Unit \\
		%\hline
		\endhead}
	\renewcommand*{\glossentry}[2]{%  Change the displayed items
		\glstarget{##1}{\sffamily\glossentryname{##1}} %
		& \sffamily\glossentrydesc{##1}% Description
		& \sffamily\glsunit{##1}  \tabularnewline
	}
}



\usepackage{smartdiagram}

%%%%%%%%%%%%%%%%%%%%%%%%%%%%%%%%%%%%%
%CUSTOMIZE TOC
%%%%%%%%%%%%%%%%%%%%%%%%%%%%%%%%%%%%%

%Differrent font in TOC
\usepackage[titles,subfigure]{tocloft} % Change font of Chapters,  LOF and LOT

\setlength{\cftbeforechapskip}{2ex}
\setlength{\cftbeforesecskip}{0.8ex}
\setlength{\cftbeforesubsecskip}{0.8ex}

\renewcommand{\cftchapfont}{%
	\sffamily\bfseries
}
\renewcommand{\cftchappagefont}{\sffamily}
\renewcommand{\cftsecfont}{\sffamily}
\renewcommand{\cftsecpagefont}{\sffamily}
\renewcommand{\cftsubsecfont}{\sffamily}
\renewcommand{\cftsubsecpagefont}{\sffamily}


\renewcommand\cftloftitlefont{\sffamily}
\renewcommand\cftfigfont{\sffamily}
\renewcommand\cftfigpagefont{\sffamily}

\renewcommand\cftlottitlefont{\sffamily}
\renewcommand\cfttabfont{\sffamily}
\renewcommand\cfttabpagefont{\sffamily}


% TOC depth
%\setcounter{secnumdepth}{4}


%%%%%%%%%%%%%%%%%%%%%%%%%%%%%%%%%%%%%
%CUSTOM PAGE LAYOUT
%%%%%%%%%%%%%%%%%%%%%%%%%%%%%%%%%%%%%

%Code for Headers and Footers adapted from http://www.kfiles.de

\newlength{\marginWidth}
\setlength\marginWidth{\marginparwidth+\marginparsep}
\newlength{\fulllinewidth}
\setlength\fulllinewidth{\textwidth+\marginWidth}

\usepackage{truncate} %Um zu lange Kapiteltitel abzuschneiden

\footskip=1.6cm
\makeatletter % = mache @ letter

%Vordefinition mehrfachverwendeter Teile
\def\oddfootSTANDARD{
	\renewcommand{\@oddfoot}{
		\hbox to\textwidth{\vbox{\hbox to\textwidth{
					\hfill
					\strut
					\hspace{1pt}
		}}}
		\hbox to\marginWidth{\vbox{\hbox to\marginWidth{
					\strut %unsichtbares Zeichen
					%               \large %Größe der Seitenzahl
					\hspace{5pt}
					\vrule width 1pt height 1cm
					\hspace{8pt}
					\textsf{\thepage}
					\hfill
		}}}\hss
	}
}

\def\evenfootSTANDARD{
	\renewcommand{\@evenfoot}{
		\hspace{-\marginWidth}
		\hbox to\marginWidth{\vbox{\hbox to\marginWidth{
					%         \large
					\strut %unsichtbares Zeichen
					\hfill
					\textsf{\thepage}
					\hspace{5pt}
					\vrule width 1pt height 1cm
					\hspace{7pt}
		}}}\hss
	}
}


%Standardstil für die gesamte Dissertation
\newcommand{\ps@thesis}{
	\renewcommand{\@oddhead}{
		\hbox to\textwidth{\vbox{\hbox to\textwidth{
					\textsf
					\hfill
					\rightmark
					\strut
					\hspace{1pt}
		}}}
		\hbox to\marginWidth{\vbox{\hbox to\marginWidth{
					\strut %unsichtbares Zeichen
					\hspace{5pt}
					\vrule width 1pt
					\hspace{5pt}
					\textsf{\thesection}  %%Zuständig für Nummerierung rechts oben; thesection produziert X.0, X.1
					\hfill
		}}}\hss
	}

	\renewcommand{\@evenhead}{
		\hspace{-\marginWidth}
		\hbox to\marginWidth{\vbox{\hbox to\marginWidth{
					\hfill
					\strut %unsichtbares Zeichen
					\textbf{\textsf{Chapter~\thechapter}}
					\hspace{5pt}
					\vrule width 1pt
					\hspace{7pt}
					\strut
		}}}\hss

		\hbox to\textwidth{\vbox{\hbox to\textwidth{
					\strut %unsichtbares Zeichen
					\truncate{.9\textwidth}{\leftmark}
					\hfill
		}}}\hss
	}

	\oddfootSTANDARD
	\evenfootSTANDARD
}


%Der PLAIN-Style der Chapter- und Sonderseiten muss redefiniert werden.
\renewcommand{\ps@plain}{
	\let\@oddhead\@empty
	\let\@evenhead\@empty
	\let\@evenfoot\@empty
	\oddfootSTANDARD
}

%Spezieller Stil für Inhaltsverzeichnis und Literaturverzeichnis (ohne Nummern wie 0.0 oder B.0)
\newcommand{\ps@thesisINTRO}{
	\renewcommand{\@oddhead}{
		\hbox to\textwidth{\vbox{\hbox to\textwidth{
					\textsf
					\hfill
					\textsf{\rightmark}
					\strut
					\hspace{1pt}
		}}}
		\hbox to\marginWidth{\vbox{\hbox to\marginWidth{
					\strut %unsichtbares Zeichen
					\hspace{5pt}
					\vrule width 1pt
					\hspace{5pt}
					\textsf{\thechapter}  %%Zuständig für Nummerierung rechts oben; thesection produziert X.0, X.1
					\hfill
		}}}\hss
	}

	\renewcommand{\@evenhead}{
		\hspace{-\marginWidth}
		\hbox to\marginWidth{\vbox{\hbox to\marginWidth{
					\hfill
					\strut %unsichtbares Zeichen
					\textbf{\textsf{Chapter}}
					\hspace{5pt}
					\vrule width 1pt
					\hspace{7pt}
					\strut
		}}}\hss

		\hbox to\textwidth{\vbox{\hbox to\textwidth{
					\strut %unsichtbares Zeichen
					\truncate{.9\textwidth}{\leftmark}
					\hfill
		}}}\hss
	}

	\oddfootSTANDARD
	\evenfootSTANDARD
}


\newcommand{\ps@thesisAPPENDIX}{
	\renewcommand{\@oddhead}{
		\hbox to\textwidth{\vbox{\hbox to\textwidth{
					\textsf
					\hfill
					\rightmark
					\strut
					\hspace{1pt}
		}}}
		\hbox to\marginWidth{\vbox{\hbox to\marginWidth{
					\strut %unsichtbares Zeichen
					\hspace{5pt}
					\vrule width 1pt
					\hspace{5pt}
					\textsf
					\thechapter %\\thesection  %%Zuständig für Nummerierung rechts oben; thesection produziert X.0, X.1
					\hfill
		}}}\hss
	}

	\renewcommand{\@evenhead}{
		\hspace{-\marginWidth}
		\hbox to\marginWidth{\vbox{\hbox to\marginWidth{
					\hfill
					\strut %unsichtbares Zeichen
					\textbf{\textsf{Appendix~\thechapter}}
					\hspace{5pt}
					\vrule width 1pt
					\hspace{7pt}
					\strut
		}}}\hss

		\hbox to\textwidth{\vbox{\hbox to\textwidth{
					\strut %unsichtbares Zeichen
					\truncate{.9\textwidth}{\leftmark}
					\hfill
		}}}\hss
	}

	\oddfootSTANDARD
	\evenfootSTANDARD
}


\newcommand{\ps@thesisLISTS}{
	\renewcommand{\@oddhead}{
		\hbox to\textwidth{\vbox{\hbox to\textwidth{
					\textsf
					\hfill
					\textsf{\rightmark}
					\strut
					\hspace{1pt}
		}}}
		\hbox to\marginWidth{\vbox{\hbox to\marginWidth{
					\strut %unsichtbares Zeichen
					\hspace{5pt}
					\vrule width 1pt
					\hspace{5pt}
					\textsf
					\thechapter  %%Zuständig für Nummerierung rechts oben; thesection produziert X.0, X.1
					\hfill
		}}}\hss
	}

	\renewcommand{\@evenhead}{
		\hspace{-\marginWidth}
		\hbox to\marginWidth{\vbox{\hbox to\marginWidth{
					\hfill
					\strut %unsichtbares Zeichen
					\textbf{\textsf{Chapter}}
					\hspace{5pt}
					\vrule width 1pt
					\hspace{7pt}
					\strut
		}}}\hss

		\hbox to\textwidth{\vbox{\hbox to\textwidth{
					\strut %unsichtbares Zeichen
					%	\truncate{.9\textwidth}{\textsf{\MakeUppercase{\leftmark}}}% Zuständig für Variable hinter |
					\truncate{.9\textwidth}{\textsf{\leftmark}}% Zuständig für Variable hinter |
					\hfill
		}}}\hss
	}

	\oddfootSTANDARD
	\evenfootSTANDARD
}




\newcommand{\ps@reallyempty}{
	\let\@oddhead\@empty
	\let\@evenhead\@empty
	\let\@oddfoot\@empty
	\let\@evenfoot\@empty
}


\renewcommand{\chaptermark}[1]{\markboth{\textsf{#1}}{}}%markboth hat zwei argumente für die linke und rechte seite
\renewcommand{\sectionmark}[1]{\markright{\textsf{#1}}}

\makeatother % = mache @ wieder zu nicht-Buchstaben
\pagestyle{thesis}



%%Problem mit den Seitenzahlen und Headern auf leeren Seiten nach Kapiteln:
\let\origdoublepage\cleardoublepage
\newcommand{\clearemptydoublepage}{%
	\clearpage
	{\pagestyle{empty}\origdoublepage}%
}
\let\cleardoublepage\clearemptydoublepage




%%%%%%%%%%%%%%%%%%%%%%%%%%%%%%%%%%%%%
%INDEX
%%%%%%%%%%%%%%%%%%%%%%%%%%%%%%%%%%%%%

\usepackage{makeidx}
\makeindex
\usepackage[totoc,columns=2,minspace=100pt]{idxlayout} %modify layout of index



%%%%%%%%%%%%%%%%%%%%%%%%%%%%%%%%%%%%%
%HELPER FOR FORMATTING GRAPHICS
%https://www.queryxchange.com/q/24_86356/how-to-trim-clip-crop-graphics-without-trial-and-error/
%%%%%%%%%%%%%%%%%%%%%%%%%%%%%%%%%%%%%


%\newcommand{\showgrid}[2]{%
%	\newcommand{\gridlen}{2}  % This determines the size of the squares that later appear: 2 = 0.4 cm % 5 = 1 cm  % 20 = 4 cm
%	\resizebox{#1}{!}{%
%		\begin{tikzpicture}[inner sep=0]
%		% Bild laden
%		\node[anchor=south west] (image) at (0, 0) {#2};
%		% Koordinaten fast oben rechts
%		\path (image.north east) -- ++(-\gridlen, -\gridlen) coordinate (obenrechts);
%
%		\begin{scope}[red]
%		% Gitter unten links
%		\draw[xstep=.2, ystep=.2, very thin] (0, 0) grid (\gridlen, \gridlen);
%		\draw[xstep=1, ystep=1, semithick] (0, 0) grid (\gridlen, \gridlen);
%		% Gitter oben rechts
%		\draw[xstep=.2, ystep=.2, shift={(obenrechts)}, very thin] (0, 0) grid (\gridlen, \gridlen);
%		\draw[xstep=1, ystep=1, shift={(obenrechts)}, semithick] (0, 0) grid (\gridlen, \gridlen);
%
%		% Rahmen
%		\draw (0, 0) rectangle (image.north east);
%		\end{scope}
%		\end{tikzpicture}%
%	}
%}

%Example:
%\showgrid{0.8\linewidth}{\includegraphics[clip, trim=31mm 58mm 102mm 31mm]{Test.pdf}}

\usepackage{tikz}

% Linen über Graphiken
\newcommand{\showgrid}[3][5]{%
	\providecommand{\griddepth}{#1}
	\resizebox{#2}{!}{%
		\begin{tikzpicture}[inner sep=0]
		% Bild laden
		\node[anchor=south west] (image) at (0, 0) {#3};
		% Linien einfügen
		\begin{scope}[red]
		% Äußere Schleife für dicke Rechtecke
		\foreach \iThick in {0, ..., \griddepth} {%
			\path (image.north east) ++(-\iThick, -\iThick) coordinate(topright);
			\draw[semithick] (\iThick, \iThick) rectangle (topright);
			% Zwischen den Linien auffüllen
			\ifnum\iThick<\griddepth
			% dünne Rechtecke
			\foreach \iThin in {1, ..., 4} {%
				\path (image.north east) ++(-\iThick, -\iThick) ++(-\iThin/5, -\iThin/5) coordinate(topright);
				\draw[very thin] (\iThick, \iThick) ++(\iThin/5, \iThin/5) rectangle (topright);
			}
			\fi
		}
		\end{scope}
		\end{tikzpicture}
	}
}

%Example:
%\showgrid[1]{0.8\linewidth{\includegraphics[clip, trim=20mm 34mm 8mm 16mm]{Test.pdf}}
%
%The thick lines have a distance of 10mm, the thin ones of 2mm. This is independent of the image if no width or height argument is passed to the image.



%%%%%%%%%%%%%%%%%%%%%%%%%%%%%%%%%%%%%
%LATEX OVERLAY GENERATOR
%%%%%%%%%%%%%%%%%%%%%%%%%%%%%%%%%%%%%

%%%%%%%%%%%%%%%%%%%%%%%%%%%%%%%%%%%%%
%LATEX OVERLAY GENERATOR
%%%%%%%%%%%%%%%%%%%%%%%%%%%%%%%%%%%%%

%%%%%%%%%%%%%%%%%%%%%%%%%%%%%%%%%%%%%%%%%%%%%%%%%%%%%%%%%%%%%%%%%%%%%%
% LaTeX Overlay Generator - Annotated Figures v0.0.1
% Created with http://ff.cx/latex-overlay-generator/
% If this generator saves you time, consider donating 5,- EUR! :-)
%%%%%%%%%%%%%%%%%%%%%%%%%%%%%%%%%%%%%%%%%%%%%%%%%%%%%%%%%%%%%%%%%%%%%%
%\annotatedFigureBoxCustom{bottom-left}{top-right}{label}{label-position}{box-color}{label-color}{border-color}{text-color}
\newcommand*\annotatedFigureBoxCustom[8]{\draw[#5,thick,rounded corners] (#1) rectangle (#2);\node at (#4) [fill=#6,thick,shape=circle,draw=#7,inner sep=2pt,font=\sffamily,text=#8] {\textbf{#3}};}
%\annotatedFigureBox{bottom-left}{top-right}{label}{label-position}
\newcommand*\annotatedFigureBox[4]{\annotatedFigureBoxCustom{#1}{#2}{#3}{#4}{white}{white}{black}{black}}
\newcommand*\annotatedFigureText[4]{\node[draw=none, anchor=south west, text=#2, inner sep=0, text width=#3\linewidth,font=\sffamily] at (#1){#4};}

\newcommand*\annotatedFigureBoxCustomBlack[8]{\draw[#5,thick,rounded corners] (#1) rectangle (#2);\node at (#4) [fill=#6,thick,shape=circle,draw=#7,inner sep=2pt,font=\sffamily,text=#8] {\textbf{#3}};}
%\annotatedFigureBox{bottom-left}{top-right}{label}{label-position}
\newcommand*\annotatedFigureBoxBlack[4]{\annotatedFigureBoxCustomBlack{#1}{#2}{#3}{#4}{black}{white}{black}{black}}
\newcommand*\annotatedFigureTextBlack[4]{\node[draw=none, anchor=south west, text=#2, inner sep=0, text width=#3\linewidth,font=\sffamily] at (#1){#4};}

\newcommand*\annotatedFigureBoxCustomGray[8]{\draw[#5,thick,rounded corners] (#1) rectangle (#2);\node at (#4) [fill=#6,thick,shape=circle,draw=#7,inner sep=2pt,font=\sffamily,text=#8] {\textbf{#3}};}
%\annotatedFigureBox{bottom-left}{top-right}{label}{label-position}
\newcommand*\annotatedFigureBoxGray[4]{\annotatedFigureBoxCustomBlack{#1}{#2}{#3}{#4}{gray}{white}{gray}{black}}
\newcommand*\annotatedFigureTextGray[4]{\node[draw=none, anchor=south west, text=#2, inner sep=0, text width=#3\linewidth,font=\sffamily] at (#1){#4};}

\newenvironment {annotatedFigure}[1]{\centering\begin{tikzpicture}
	\node[anchor=south west,inner sep=0] (image) at (0,0) { #1};\begin{scope}[x={(image.south east)},y={(image.north west)}]}{\end{scope}\end{tikzpicture}}
%%%%%%%%%%%%%%%%%%%%%%%%%%%%%%%%%%%%%%%%%%%%%%%%%%%%%%%%%%%%%%%%%%%%%% 





%%%%%%%%%%%%%%%%%%%%%%%%%%%%%%%%%%%%%
%CLEAR DOUBLE PAGE PROPERLY
%%%%%%%%%%%%%%%%%%%%%%%%%%%%%%%%%%%%%

%Input something on the left page, use \cleardoublepage to force on right
\makeatletter
\newcommand*{\cleartoleftpage}{%
	\clearpage
	\if@twoside
	\ifodd\c@page
	\hbox{}\newpage
	\if@twocolumn
	\hbox{}\newpage
	\fi
	\fi
	\fi
}
\makeatother


\usepackage[percent]{overpic}
%%%%%%%%%%%%%%%%%%%%%%%%%%%%%%%%%%%%%
%Fix warnings
%%%%%%%%%%%%%%%%%%%%%%%%%%%%%%%%%%%%%
\usepackage{silence}\WarningFilter{fixltx2e}{}
